%%%%%%%%%%%%%%%%%%%%%%%%%%%%%%%%%%%%%%%%%%%%%%%%%%%%%%%%%%%%%%%%%%%%%%
% LaTeX Template: Curriculum Vitae
%
% Source: http://www.howtotex.com/
% Feel free to distribute this template, but please keep the
% referal to HowToTeX.com.
% Date: July 2011
% 
%%%%%%%%%%%%%%%%%%%%%%%%%%%%%%%%%%%%%%%%%%%%%%%%%%%%%%%%%%%%%%%%%%%%%%
% How to use writeLaTeX: 
%
% You edit the source code here on the left, and the preview on the
% right shows you the result within a few seconds.
%
% Bookmark this page and share the URL with your co-authors. They can
% edit at the same time!
%
% You can upload figures, bibliographies, custom classes and
% styles using the files menu.
%
% If you're new to LaTeX, the wikibook is a great place to start:
% http://en.wikibooks.org/wiki/LaTeX
%
%%%%%%%%%%%%%%%%%%%%%%%%%%%%%%%%%%%%%%%%%%%%%%%%%%%%%%%%%%%%%%%%%%%%%%
\documentclass[paper=a4,fontsize=11pt]{scrartcl} % KOMA-article class

\usepackage[english]{babel}
\usepackage[utf8x]{inputenc}
\usepackage[protrusion=true,expansion=true]{microtype}
\usepackage{amsmath,amsfonts,amsthm}     % Math packages
\usepackage{graphicx}                    % Enable pdflatex
\usepackage[svgnames]{xcolor}            % Colors by their 'svgnames'
\usepackage[hidelinks, colorlinks=true, urlcolor=blue]{hyperref}
\usepackage[
left=0.4in,
right=0.4in,
top=0.6in,
bottom=0.4in]{geometry}

\usepackage{url}

\frenchspacing              % Better looking spacings after periods
\pagestyle{empty}           % No pagenumbers/headers/footers


%%% Custom sectioning (sectsty package)
%%% ------------------------------------------------------------
\usepackage{sectsty}

\sectionfont{%			            % Change font of \section command
	\usefont{OT1}{phv}{b}{n}%		% bch-b-n: CharterBT-Bold font
	\sectionrule{0pt}{0pt}{-5pt}{1pt}}

%%% Macros
%%% ------------------------------------------------------------
\newlength{\spacebox}
\settowidth{\spacebox}{8888888888}			% Box to align text
\newcommand{\sepspace}{\vspace*{0.5em}}		% Vertical space macro

\newcommand{\MyName}[1]{ % Name
	\Huge \usefont{OT1}{phv}{b}{n} \begin{center}#1\end{center}
	\par \normalsize \normalfont \vspace*{-1.5em}}

\newcommand{\MySlogan}[1]{ % Slogan (optional)
	\large \usefont{OT1}{phv}{m}{n} \begin{center}#1\end{center}
	\par \normalsize \normalfont}

\newcommand{\NewPart}[1]{\section*{\uppercase{#1}}}

\newcommand{\PersonalEntry}[2]{
	\noindent\hangindent=2em\hangafter=0 % Indentation
	\parbox{\spacebox}{        % Box to align text
		\textit{#1}}		       % Entry name (birth, address, etc.)
	\hspace{1.5em} #2 \par}    % Entry value

\newcommand{\SkillsEntry}[2]{      % Same as \PersonalEntry
	\noindent\hangindent=1em\hangafter=0 % Indentation
	\parbox{100pt}{	        % Box to \align text
		\textit{#1}}			   % Entry name (birth, address, etc.)
	\hspace{1.5em} #2 \par}    % Entry value	

\newcommand{\EducationEntry}[3]{
	\noindent \textbf{#1} \hfill      % Study
	\colorbox{White}{%
		\parbox{8em}{%
			\hfill\color{Black}\textbf{#2}}} \par  % Duration
	\small #3 % Description
	\normalsize \par}


\newcommand{\SomeEntry}[4]{
	\noindent \textbf{#1} \hfill      % Study
	\colorbox{White}{%
		\parbox{12em}{%
			\hfill\color{Black}\textbf{#2}}} \par  % Duration
	\noindent \texttt{#3} \par        % School
	\noindent\hangindent=2em\hangafter=0 \small #4 % Description
	\normalsize \par}

\begin{document}
	
	\MyName{Vasily Kuporosov}
	\MySlogan{vasyoid@gmail.com\qquad\href{http://github.com/vasyoid}{github.com/vasyoid}\qquad+7(921)437-35-21}
	
	\NewPart{Education}
	
	\EducationEntry{Higher School of Economics}{2020 - present}{Master's degree in Programming and Data Analysis}
	
	\EducationEntry{Higher School of Economics}{2018 - 2020}{Bachelor's degree in Mathematics and Computer Science (years 3-4) \textbf{GPA 9.27/10.0}}
	
	\EducationEntry{Saint Petersburg Academic University}{2016 - 2018}{Bachelor’s degree in Mathematics and Computer Science (years 1-2) }
	
	\NewPart{Work Experience}
	
	\SomeEntry{JetBrains – Software developer in CoSpaces project}{Feb 2018 - present}{\vspace{-10pt}}{Implemented a triangulation module for complex polygons with holes;\\
		Created a servlet that converts a text into Bézier curves according to a specified font;\\
		Implemented a 3D-text builder that takes a text, triangulates its curves and extrudes them to create a 3D item;\\
		3D text is now used in CoSpaces Edu – a web app which allows creating 3D spaces for educational purposes.\\
		Technologies used: Kotlin for server side, Java for client-side, TeaVM to compile Java into JavaScript.}
	
	\SomeEntry{Yandex QA automation internship}{Jul-Oct 2018}{\vspace{-10pt}}{Implemented ∼200 test cases, improved testing system by optimizing its components (e.g. parallel message sending);
		Used Selenium WebDriver, Selenium Maven project for java and JUnit testing framework.}
	
	\NewPart{Projects}
	
	\SomeEntry{Java + RISC-V (team project, bachelor's thesis)}{Oct 2019 - June 2020}{\href{https://github.com/azul-research/jdk-riscv}{github.com/azul-research/jdk-riscv}}{OpenJDK port for RISC-V processor architecture. The goal of the project is to implement a platform-specific part of OpenJDK so that it can run on RISC-V processors.\\
		Implemented all arithmetic bytecode instructions and several math intrinsic functions in bytecode interpreter;\\
		Contributed in some other parts of the interpreter.}
	\sepspace
	\SomeEntry{Testopithecus (team project)}{Mar-May 2019}{\href{https:/github.com/smart-testing/testopithecus}{github.com/smart-testing/testopithecus}}{A research on how to make automated testing smarter. The prototype has been adopted by the Yandex.mail team.\\
		Created an infrastructure for benchmarking crawling and crash-seeking algorithms on Android, iOS and web;\\
		Invented and implemented various enhancements for automated UI interaction;\\
		Implemented a tool that reconstructs an application usage scenario based on debug logging output.} 
	\sepspace
	\SomeEntry{Automated image dataset segmentation tool}{Feb-May 2018}{\href{https://github.com/vasyoid/segmentation}{github.com/vasyoid/segmentation}}{An algorithm that automatically segments a dataset of images to an object and background. A user roughly marks an object and background on the first image, after that the precise segmentation of the image is built and then spread to the whole dataset.} 
	
	\NewPart{Knowledge and Skills}
	
	\SkillsEntry{Used in projects}{C, C++, Java, Kotlin, Android SDK, Python, Algorithms and Data Structures.}
	\SkillsEntry{Classroom experience}{Unity, Scala, \LaTeX, Docker, Git.}
\end{document}
